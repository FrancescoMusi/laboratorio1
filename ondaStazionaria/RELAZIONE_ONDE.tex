\documentclass[12pt, a4paper]{article}
\usepackage{graphicx}
\usepackage{mathtools}
\usepackage{xcolor}
\usepackage{amsmath}
\usepackage{caption}
\usepackage[italian]{babel}
\usepackage{eso-pic}
\usepackage{setspace}
\usepackage{multirow}
\usepackage{array}
\usepackage{geometry}
\usepackage{longtable}
\graphicspath{ {./immagini/} }
\linespread{1.1}
\binoppenalty=10000 %impedisce di separare a capo formule matematiche nel testo
\relpenalty=10000
\geometry{
    total={170mm,257mm},
    left=20mm,
    top=15mm,
    bottom=28mm
}

 
\title{\textbf{Misura della velocità di fase di un’onda
elastica trasversale in una fune confinata}}
\date{}

 
\begin{document}
\maketitle
\AddToShipoutPictureBG*{%
  \AtPageUpperLeft{%
    \hspace{\paperwidth}%
    \raisebox{-\baselineskip}{%
      \makebox[0pt][r]{\textbf{Gruppo 10}: Mussini Simone, Musi Francesco, Ruscillo Fabio        }
}}}%



\section{Richiami teorici}
Un'onda è una perturbazione che viene generata da una sorgente. È caratterizzata da lunghezza d'onda ($\lambda \ [m]$), ampiezza ($A\  [m]$) e frequenza ($f \ [s^{-1}]$).
Essa può essere un'onda meccanica che si propaga attraverso un mezzo, determinando così uno spostamento di particelle, oppure può essere un'onda elettromagnetica, che può propagarsi anche nel vuoto essendo una perturbazione di un campo elettrico e magnetico.
\bigskip

Un vincolo o una superficie, possono causare il fenomeno di riflessione di un'onda. 
Se l'onda riflessa si propaga nella stessa direzione dell'onda iniziale, ma con verso opposto come nel nostro caso, si genera un\textit{ fenomeno di interferenza} tra l'onda di partenza e quella riflessa.
\bigskip

 L'interferenza può dar luogo ad un'onda stazionaria se essa "non si propaga", ma rimane confinata nella stessa zona di spazio.  Ogni punto si comporta come un oscillatore armonico, ad eccezione di alcuni punti fissi detti \textit{nodi}, la cui quota non varia mai.
 
Segue l'equazione di un'onda stazionaria:

\begin{equation*}
    y = \frac{A}{2} sin(kx - \omega t) + \frac{A}{2} sin(kx + \omega t) = A\  sin(kx)cos(\omega t)
\end{equation*}




\addvspace{2cm}
\section{Obbiettivo}
Lo scopo di questa esperienza, è quello di calcolare la velocità di propagazione di un'onda elastica trasversale, attraverso la misura della frequenza e della lunghezza d'onda di varie onde stazionarie, ottenute per diversi modi.

\newpage

\section{Materiale}
\begin{itemize}
\setlength\itemsep{0mm}
    \item Fune omogenea
    \item Sostegni
    \item Peso da 5 $kg$
    \item Generatore di segnale con pistone
    \item Metro a nastro da 2,10 $m$
\end{itemize}



\addvspace{2cm}
\section{Procedimento}
Il set-up è montato in modo che ci sia una corda \textit{tesa} legata ad un sostegno fisso da un lato, mentre dall'altro, fissata ad un peso da $5$ $kg$. Il pistone è stato posto in prossimità dell'estremo fisso, in modo tale da limitare l'errore sulla misura della distanza tra i  nodi.


Tramite il generatore di segnale, si è aumentata progressivamente la frequenza $f$, fino ad osservare un'onda stazionaria. Per ogni valore di $f$ per cui ciò accade, si misura: 
\begin{itemize}
    \item la distanza tra due nodi successivi: $\quad d_n$
    \item la larghezza di ogni nodo: $\quad\Delta d_n$
\end{itemize}

Poichè in condizioni ideali i nodi sono puntiformi, si è deciso di attribuire la misura della larghezza dei nodi, come errore sulla distanza tra due nodi successivi.




  \addvspace{2cm}
\subsection{Frequenza e distanza tra gli internodi}
Per avere una prima stima delle frequenze di risonanza, si è acceso il generatore di segnale collegato al pistone, il quale ha cominciato a generare delle oscillazioni sulla corda. 
\bigskip

Si è iniziato a modulare la frequenza e l'intensità del segnale, fino ad ottenere un'onda stazionaria con ampiezza maggiore possibile.
\bigskip

Dopo aver ripetuto l'operzione ottenendo diversi modi, si è calcolata con maggiore precisione la frequenza di risonanza del modo fondamentale, i cui multipli permettono di individuare il range delle frequenze dei modi successivi, considerando come errore sulle frequenze quello strumentale, poichè è l'unica fonte di errore rilevante.
 \bigskip
 
Infine si è calcolata la distanza tra due nodi successivi e l'errore su ogni singolo nodo, così da poter effettuare una misura più accurata sull'errore da attribuire alla misura degli internodi, stando attenti nell'assegnare un errore maggiore agli estremi della corda, poichè i nodi, sono meno individuabili.


Sono state riportate sia le frequenze che le distanze tra due nodi successivi, con i rispettivi errori, in Tabella \ref{tab: Frequenza e distanza } (vedi Colonna \begin{footnotesize}{\textit{(2)}}\end{footnotesize}: frequenza, Colonna \begin{footnotesize}{\textit{(3)}}\end{footnotesize}: distanza tra due nodi) 

\newpage
\begin{table}[!htb]
\centering
\begin{tabular}{|c|c|c|}
\hline
    \textbf{Modo} & \textbf{Frequenza} & \textbf{Distanza tra due nodi} \\ 
    \footnotesize$(n)$ & \footnotesize$(f\pm\Delta f)$($s^{-1}$) & \footnotesize$(d_n\pm\Delta d_n)$ $(m)$ \\ 
    \hline\hline
    \footnotesize$1$ & \footnotesize$7.0\pm0.1$ & \footnotesize$2.007 \pm 0.015$ \\[0.7ex] 
    \hline %freq          %dist.inter.
    \footnotesize$2$ & \footnotesize$14.0 \pm 0.1 $ &  \footnotesize$1.005 \pm 0.008$ \\[0.7ex]
    \footnotesize & \footnotesize & \footnotesize$1.005 \pm 0.013$ \\[0.7ex]
    \hline
    \footnotesize & \footnotesize & \footnotesize$0.665 \pm 0.008$ \\ [0.7ex]
    \footnotesize$3$ & \footnotesize$20.9 \pm 0.1$ & \footnotesize$0.670 \pm 0.005$  \\ [0.7ex]
    \footnotesize & \footnotesize & \footnotesize$0.680 \pm 0.012$ \\ [0.7ex]
    \hline 
    \footnotesize & \footnotesize & \footnotesize$0.500 \pm 0.008$ \\ [0.7ex]
    \footnotesize$4$ & \footnotesize$27.8 \pm 0.1$ & \footnotesize$0.508 \pm 0.005$ \\ [0.7ex]
    \footnotesize & \footnotesize & \footnotesize$0.505 \pm 0.005$ \\ [0.7ex]
    \footnotesize & \footnotesize & \footnotesize$0.505 \pm 0.013$ \\ [0.7ex]
    \hline
    \footnotesize & \footnotesize & \footnotesize$0.401 \pm 0.008$ \\ [0.7ex]
    \footnotesize & \footnotesize & \footnotesize$0.411 \pm 0.005$ \\ [0.7ex]
    \footnotesize$5$ & \footnotesize$34.7 \pm 0.1$ & \footnotesize$0.407 \pm 0.004$ \\ [0.7ex]
    \footnotesize & \footnotesize & \footnotesize$0.395 \pm 0.005$ \\ [0.7ex]
    \footnotesize & \footnotesize & \footnotesize$0.407 \pm 0.013$ \\ [0.7ex]
    \hline
    \footnotesize & \footnotesize & \footnotesize$0.330 \pm 0.006$ \\ [0.7ex]
    \footnotesize & \footnotesize & \footnotesize$0.333 \pm 0.003$ \\ [0.7ex]
    \footnotesize$6$ & \footnotesize$41.7 \pm 0.1$ & \footnotesize$0.336 \pm 0.005$ \\ [0.7ex]
    \footnotesize & \footnotesize & \footnotesize$0.335 \pm 0.004$ \\ [0.7ex]
    \footnotesize & \footnotesize & \footnotesize$0.332 \pm 0.002$ \\ [0.7ex]
    \footnotesize & \footnotesize & \footnotesize$0.339 \pm 0.011$ \\ [0.7ex]
    \hline
    \footnotesize & \footnotesize & \footnotesize$0.288 \pm 0.006$ \\ [0.7ex]
    \footnotesize & \footnotesize & \footnotesize$0.288 \pm 0.003$ \\ [0.7ex]
    \footnotesize & \footnotesize & \footnotesize$0.291 \pm 0.004$ \\ [0.7ex]
    \footnotesize$7$ & \footnotesize$48.7 \pm 0.1$ & \footnotesize$0.285 \pm 0.004$ \\ [0.7ex]
    \footnotesize & \footnotesize & \footnotesize$0.277 \pm 0.004$ \\ [0.7ex]
    \footnotesize & \footnotesize & \footnotesize$0.282 \pm 0.003$ \\ [0.7ex]
    \footnotesize & \footnotesize & \footnotesize$0.290 \pm 0.011$ \\ [0.7ex]
    \hline

    \hline
 \end{tabular}
 \caption{\small{\textit{Colonna \begin{footnotesize}{\textit{(1)}}\end{footnotesize}: numero di modi \textbf{-} Colonna \begin{footnotesize}{\textit{(2)}}\end{footnotesize}: frequenza \textbf{-} 
 Colonna \begin{footnotesize}{\textit{(3)}}\end{footnotesize}: distanza tra i nodi}}}
 \label{tab: Frequenza e distanza }
 \end{table} 

\subsection{Distanza media degli tra due nodi successivi e Lunghezza d'onda}
Poichè la distanza tra due nodi e il rispettivo errore, variano da internodo a internodo, si è scelto di utilizzare la media pesata per il calcolo della distanza media e dell'errore medio.

Per il calcolo sulla distanza media si usa la formula:
\begin{equation*}
\begin{aligned}
  &\overline{d} =\frac{\sum w_n\cdot d_n}{\sum w_n}
  &\quad{} 
  \end{aligned}
  &
  \begin{aligned}
  & & & & & & & & &\text{dove il peso di ogni misura è:}& w_n=\frac{1}{(\Delta d_n)^2}
  &
  \end{aligned}
\end{equation*}
Invece l'errore medio è calcolato come:
\begin{equation*}
    \overline{\Delta{d}} =\frac{1}{\sqrt{\sum w_n}}
\end{equation*}

\addvspace{1.5cm}
Si ottengono quindi i seguenti valori:
{
\renewcommand\arraystretch{1.2} %spaziatura tra le colonne (da 1 a 1.1)

\begin{table}[ht] %[hl] serve per mettere la tabella nel mezzo del testo


\begin{tabular}{|c|c|c|c|c|c|c|c|} 
 
 \hline
  \small Modi (n) & 1 & 2 & 3 & 4 & 5 & 6 & 7\\
  
\hline

  
  \small $\overline{d}\pm\overline{\Delta{d}}$ ($m$) &\footnotesize{2.007$\pm$0.015}  &\footnotesize{1.005$\pm$0.007} &\footnotesize{0.670$\pm$0.004}&\footnotesize{0.505$\pm$0.003}&\footnotesize{0.404$\pm$0.002} &\footnotesize{0.333$\pm$0.001}&\footnotesize{0.285$\pm$0.001}\\
\hline


\end{tabular}\\
\caption{\small{\textit{Misura della distanza media tra gli internodi, con errore ottenuto tramite media pesata.} }}
    \label{tab:Error_MediapesataFalse}
\end{table}
}

\addvspace{1cm}
L'errore ottenuto tramite media pesata, è \textbf{\textit{inferiore}} all'errore rilevato, quindi si considera come errore sul valor medio della distanza tra due nodi successivi $\overline{\Delta d}$, il più piccolo tra quelli misurati, ottenendo:

{
\renewcommand\arraystretch{1.2} %spaziatura tra le colonne (da 1 a 1.1)

\begin{table}[ht] %[hl] serve per mettere la tabella nel mezzo del testo


\begin{tabular}{|c|c|c|c|c|c|c|c|} 
 
 \hline
  \small Modi (n) & 1 & 2 & 3 & 4 & 5 & 6 & 7\\
  
\hline

  
  \small $\overline{d}\pm\overline{\Delta{d}}$ ($m$) &\footnotesize{2.007$\pm$0.015}  &\footnotesize{1.005$\pm$0.008} &\footnotesize{0.670$\pm$0.005}&\footnotesize{0.505$\pm$0.005}&\footnotesize{0.404$\pm$0.004} &\footnotesize{0.333$\pm$0.002}&\footnotesize{0.285$\pm$0.003}\\
\hline


\end{tabular}\\
\caption{\small{\textit{Misura della distanza media tra gli internodi, avente come errore il minore tra quelli misurati per ogni modo.} }}
    \label{tab:Error_MediapesataFalse}
\end{table}
}





Successivamente si è calcolata la lunghezza d'onda, moltiplicando per due la distanza media e l'errore sulla distanza media, poichè la distanza tra due nodi è la metà di una lunghezza d'onda:
\begin{equation*}
    \lambda=2\cdot\overline{d}
\end{equation*}
\begin{equation*}
    \Delta\lambda=2\cdot\overline{\Delta{d}}
\end{equation*}

\addvspace{1.5cm}
 I risultati ottenuti sono riportati di seguito:
 
 \addvspace{1cm}
 {
\renewcommand\arraystretch{1.2} %spaziatura tra le colonne (da 1 a 1.1)

\begin{table}[ht] %[hl] serve per mettere la tabella nel mezzo del testo


\begin{tabular}{|c|c|c|c|c|c|c|c|} 
 
 \hline
  \small Modi (n) & 1 & 2 & 3 & 4 & 5 & 6 & 7\\
  
\hline

  
  \small $\lambda\pm\Delta\lambda$ ($m$) &\footnotesize{4.014$\pm$0.030}  &\footnotesize{2.010$\pm$0.016} &\footnotesize{1.340$\pm$0.010}&\footnotesize{1.010$\pm$0.010}&\footnotesize{0.808$\pm$0.008} &\footnotesize{0.666$\pm$0.004}&\footnotesize{0.570$\pm$0.006}\\
\hline


\end{tabular}\\
\caption{\small{\textit{Lunghezza d'onda di ogni modo} }}
    \label{tab:Lunghezza_d'onda}
\end{table}
}

\addvspace{1.5cm}
\section{Grafico qualitativo in scala lineare $\lambda(f)$}
\label{sez: Graf qualitativo}
Si procede con un grafico qualitativo in scala lineare $\lambda(f)$ e si verifica il tipo di relazione che sussiste tra lunghezza d'onda e frequenza. 


I dati sono riportati nella seguente Tabella:
\begin{table}[ht] %[hl] serve per mettere la tabella nel mezzo del testo
 \centering

\begin{tabular}{|c|c|c|} 
 \hline
  \footnotesize{Modi (n)}& \footnotesize{$\lambda\pm\Delta\lambda$ ($m$)}  & \footnotesize{$f\pm \Delta f $ ($s^{-1}$)} \\ 
\hline
 $1$ & \footnotesize{$4.014\pm0.030$} & \footnotesize{$7.0\pm0.1$} \\   
 $2$ & \footnotesize{$2.010\pm0.016$} & \footnotesize{$14.0\pm0.1$} \\
 $3$ & \footnotesize{$1.340\pm0.010$} & \footnotesize{$20.9\pm0.1$} \\
 $4$ & \footnotesize{$1.010\pm0.010$} & \footnotesize{$27.8\pm0.1$} \\
 $5$ & \footnotesize{$0.808\pm0.008$} & \footnotesize{$34.7\pm0.1$} \\
 $6$ & \footnotesize{$0.666\pm0,004$} & \footnotesize{$41.7\pm0.1$} \\
 $7$ & \footnotesize{$0.570\pm0.006$} & \footnotesize{$48.7\pm0.1$} \\
\hline

\end{tabular}
\caption{\small{\textit{In Tabella sono riportati i valori della lunghezza d'onda e della frequenza coi relativi errori.} }}
    \label{tab:lambda frequenza}
\end{table}
\bigskip

Da cui si ottiene il seguente Grafico:
\begin{figure}[!htb]
\centering
\includegraphics[width=175mm]{immagini/graf qualitativo lam freq.jpg}
    \centering
    \caption{\textit{Sull'asse $y$ sono riportati i valori relativi alla lunghezza d'onda, sull'asse $x$ quelli relativi alla frequenza.}}
    \label{fig: qualitativo}
\end{figure}

Come si può notare, tra lunghezza d'onda $\lambda$ e frequenza $f$ sussiste una relazione di potenza.

Risulta quindi utile effettuare una regressione lineare.


\section{Regressione lineare}
La relazione che lega la lunghezza d'onda e la frequenza, ci si aspetta che sia una legge di potenza del tipo: 
\begin{equation*}
    \lambda = Cf^{q}
\end{equation*}
dove $q_{atteso}=-1$ con errore $\Delta q_{atteso}$ trascurabile.
Al fine di trovare i parametri $C$ e $q$, si studiano i dati tramite una regressione lineare. Passando ai logaritmi si ottiene:
\begin{equation*}
    \ln{(\lambda)}=a +b\ \ln{(f)} \quad , \quad     \Delta\ln{(\lambda)}=\frac{\Delta \lambda}{\lambda}\ 
\end{equation*}

dove 

\begin{equation*}
    ln(C) = a \quad,\quad q = b 
\end{equation*}

I valori ottenuti sono riportati nella seguente Tabella :


\begin{table}[ht] %[hl] serve per mettere la tabella nel mezzo del testo
 \centering

\begin{tabular}{|c|c|c|} 
 \hline
  \footnotesize{Modi (n)}& \footnotesize{$ln(\lambda)\pm\Delta ln(\lambda)$ }  & \footnotesize{$ln(f) \pm \Delta ln(f) $ } \\ 
\hline
 $1$ & \footnotesize{$1.390\pm0.007$} & \footnotesize{$1.946\pm0.014$}\\ 
 $2$ & \footnotesize{$0.698\pm0.008$} & \footnotesize{$2.639\pm0.007$} \\
 $3$ & \footnotesize{$0.293\pm0.007$} & \footnotesize{$3.040\pm0.005$} \\
 $4$ & \footnotesize{$0.010\pm0.010$} & \footnotesize{$3.325\pm0.004$} \\
 $5$ & \footnotesize{$-0.213\pm0.010$} & \footnotesize{$3.547\pm0.003$} \\
 $6$ & \footnotesize{$-0.406\pm0.006$} & \footnotesize{$3.731\pm0.002$} \\
 $7$ & \footnotesize{$-0.562\pm0.011$} & \footnotesize{$3.886\pm0.002$} \\
\hline

\end{tabular}\\
\caption{\small{\textit{In Tabella sono riportati i valori del logaritmo della lunghezza d'onda $ln(\lambda)$ e del logaritmo della frequenza $ln(f)$, coi rispettivi errori.} }}
    \label{tab:Logaritmo lunghezza d'onda e frequenza}
\end{table}

Il Grafico ottenuto è il seguente:

\begin{figure}[h!]
\centering
\includegraphics[width=150mm, height=83mm]{immagini/log-log.jpg}
    \centering
    \caption{\textit{Sull'asse $y$ sono riportati i valori del logaritmo della lunghezza d'onda, sull'asse $x$ quelli del logaritmo della frequenza.}}
\end{figure}

dove 
\begin{itemize}
    \item $a=ln(C)=(3.351\pm0.014)$
    \item $b=q=(-1.006\pm0.004)$
\end{itemize}

Si verifica la compatibilità di $q$ col valore atteso $q_{atteso}=-1$ e si ottiene:
\begin{equation*}
    |q-q_{atteso}| = 0.006 
\end{equation*} 
\begin{equation*}
     \Delta q + \Delta q_{attesso} = 0.004
\end{equation*}
\bigskip

Il valore $q$ ottenuto risulta non compatibile poichè $|q-q_{atteso}|>\Delta q + \Delta q_{attesso}$.
\bigskip

Questo può essere dovuto al fatto che nel calcolo dell'errore sulla distanza media tra i nodi $\overline{\Delta d}$, si è considerato il valore minore degli errori misurati, e non quello ottenuto tramite una media pesata.
\bigskip

\subsection{Grafico con confidenza al $95\%$}
Si è deciso quindi di passare ad una confidenza pari al 95$\%$, da cui si ottiene il seguente Grafico:

\begin{figure}[!htb]
\centering
\includegraphics[width=175mm]{immagini/con 95 log-log.jpg}
    \centering
    \caption{\textit{Sull'asse $y$ sono riportati i valori del logaritmo della lunghezza d'onda, sull'asse $x$ quelli del logaritmo della frequenza, con confidenza al $95\%$.}}
    \label{im: 95 log log}
\end{figure}
\newpage

dove 
\begin{itemize}
    \item $a'=ln(C)=(3.35\pm0.02)$
    \item $b'=q'=(-1.006\pm0.006)$
\end{itemize}

Si verifica la compatibilità del nuovo valore $q'$, col valore atteso $q_{atteso}=-1$:
 
\begin{equation*}
    |q'-q_{atteso}| = 0.006
\end{equation*}
\begin{equation*}
    \Delta q' + \Delta q_{atteso} = 0.006 
\end{equation*}
Il valore di $q'$ ottenuto con una confidenza del $95\%$ risulta compatibile con quello atteso, pertanto si può procedere alla misura della velocità di fase $C$.





\subsection{Velocità di fase tramite regressione linere}
\label{Vel fase reg}
Grazie al valore di $a'$ presente in Figura \ref{im: 95 log log}, è possibile calcolare la velocità di fase $C$.

Poichè $a'=\ln(C)$ si ha che:

\begin{equation*}
    C=e^{a'}= 28.5027\ m/s\quad\quad , \quad\quad \Delta C=\left |\frac{d C}{d a'}\right| \Delta a= e^{a'}\cdot \Delta a'=0.570 \approx 0.6\ m/s
\end{equation*}

Quindi la velocità di fase dell'onda è pari a:
\begin{equation*}
    C=(28.5\pm 0.6)\ \frac{m}{s}
\end{equation*}

\addvspace{2.5cm}
\section{Verifica delle unità di misura della velocità di fase}
\label{sez: verifica unità di misura}
Ci si è chiesti se effettivamente la velocità di fase $C$ fosse una velocità.

Dalla formula $\lambda=Cf^{q}$, si verifica che l'unità di misura di C corrisponde a quella di una velocità. Sia $[x]$ l'unità di misura di C incognita, $[m]$ quella della lunghezza d'onda ed $[s]^{-1}$ quella della frequenza, si avrà:

\begin{equation*}
    [m]=[x]\cdot[s]^{-1 \cdot q}\quad\quad \xrightarrow{}\quad\quad [x]=[m]\cdot [s]^q
\end{equation*}

Si noti come solo nel caso in cui $q=-1$, l'unità di misura $[x]$ di $C$ risulta essere proprio quella di una velocità.

%\addvspace{2.5cm}
\section{Velocità di fase tramite Grafico lineare $\lambda(  \frac{1}{f}) $}


Si confronta il valore della velocità di fase $C$ della Sezione \ref{Vel fase reg}, ottenuto tramite regressione lineare, con il valore ottenuto dalla pendenza della retta $\lambda=\frac{C'}{f}$ del Grafico lineare $\lambda(\frac{1}{f})$ (vedi Figura \ref{im: rel lin lam inv freq}).

Si riportano in Tabella i valori della lunghezza d'onda e dell'inverso della frequenza, 

dove $\displaystyle  \Delta \frac{1}{f}=\left| \frac{d(\frac{1}{f})}{df}\right|\ \Delta f=\frac{1}{f^2}\ \Delta f$.



\addvspace{1cm}
{

\renewcommand\arraystretch{1.08}
\begin{table}[ht] %[hl] serve per mettere la tabella nel mezzo del testo
 \centering

\begin{tabular}{|c|c|c|} 
 \hline
  {Modi (n)}& {$\lambda\pm\Delta\lambda$ ($m$)}  & {$ \frac{1}{f}\pm \Delta \frac{1}{f} $ ($s$)} \\ 
\hline
 $1$ & \footnotesize{$4.014\pm0.030$} & \footnotesize{$0.143\pm0.002$} \\   
 $2$ & \footnotesize{$2.010\pm0.016$} & \footnotesize{$0.0714\pm0.0005$} \\
 $3$ & \footnotesize{$1.340\pm0.010$} & \footnotesize{$0.0478\pm0.0002$} \\
 $4$ & \footnotesize{$1.010\pm0.010$} & \footnotesize{$0.0360\pm0.0001$} \\
 $5$ & \footnotesize{$0.808\pm0.008$} & \footnotesize{$0.02882\pm0.00008$} \\
 $6$ & \footnotesize{$0.666\pm0.004$} & \footnotesize{$0.02398\pm0.00006$} \\
 $7$ & \footnotesize{$0.570\pm0.006$} & \footnotesize{$0.02053\pm0.00004$} \\
\hline

\end{tabular}
\caption{\small{\textit{In Tabella sono riportati i valori della lunghezza d'onda e dell'inverso della frequenza coi relativi errori.} }}
    \label{tab:lambda inv frequenza}
\end{table}

}

Da cui si ottiene il Grafico seguente:
\begin{figure}[!htb]
\centering
\includegraphics[width=175mm]{immagini/Lambda inv freq.jpg}
    \centering
    \caption{\textit{Sull'asse $y$ sono riportati i valori della lunghezza d'onda, sull'asse $x$ quelli dell'inverso della frequenza.}}
    \label{im: rel lin lam inv freq}
\end{figure}

\bigskip

Il valore della pendenza $c$ della retta ottenuta, corrisponde al valore $C'$ della velocità di fase, quindi:
\begin{equation*}
    C'=(28.23\pm0.19)\ \frac{m}{s}
\end{equation*}

Il valore appena ottenuto $C'$, risulta compatibile col valore $C$ calcolato in precendeza, poichè:
\begin{equation*}
|C-C'|=0.27\ \frac{m}{s}\quad ,\quad \Delta C + \Delta C'=0.79\ \frac{m}{s}\quad\quad \xrightarrow{}\quad\quad  |C-C'|<\Delta C + \Delta C'
\end{equation*}

 
\section{Conclusioni}
Una prima difficoltà è stata quella di far si che la corda tesa, una volta azionato il pistone posto in uno degli estremi, non muovesse il fermo posizionato nell'estremo opposto, modificando così la lunghezza della fune e le frequenze per cui si osservano delle onde stazionarie, rendendo incompatibile la legge di potenza $\lambda=\frac{C}{f}$ con i dati misurati. 

Inoltre, poichè i nodi non sono ideali e puntiformi, ma estesi, è stato necessario stimare la loro estensione singolarmente per ciascuno di essi, attribuendo quest'ultima misura come errore sulla distanza tra due nodi successivi $\Delta d_n$.

\bigskip

Poichè la distanza tra due nodi successivi ed il rispettivo errore, variano da internodo ad internodo, si è attuata una media pesata, così da avere una misura media degli internodi ed un errore medio, per ciascun modo. Successivamente, tramite la misura della distanza media, è stato possibile calcolare la lunghezza d'onda $\lambda$, ed il rispettivo errore, che caratterizza ogni modo.

\bigskip

Si è scelto di effettuare un Grafico qualitativo per avere un'idea preliminare sulla legge che lega lunghezza d'onda e frequenza (vedi Sezione \textit{\ref{sez: Graf qualitativo}}). Osservando il Grafico in Figura \ref{fig: qualitativo} si è ipotizzato che sussista una legge di potenza del tipo $\lambda=Cf^q$, dove $C$ e $q$ sono da determinare.

\bigskip

Si è proseguito con una regressione lineare, quindi si è passati ai logaritmi ottenendo la relazione:
\begin{center}
    $\phantom{aaaaa}\displaystyle ln(\lambda)=ln(C)+q\ ln(f)$
\end{center}
dove$\quad$ $a=ln(C)$ $\quad$e$\quad$ $b=q$.
\bigskip

Il valore ottenuto $b=q$ non è risultato compatibile col valore atteso $q_{atteso}=-1$, questo forse può essere dovuto al fatto che nel calcolo dell'errore sulla distanza media tra due internodi $\overline{\Delta d}$, l'errore ottenuto tramite media pesata, risulta inferiore a quello rilevato, per cui si è scelto di considerare quello più piccolo misurato come errore su $\overline{d}$.
\bigskip

Si è passati al calcolo di $q$ con una confidenza al $95\%$ (vedi Figura \textit{\ref{im: 95 log log}}), ottenendo così i valori:
\begin{center}
   $a'=ln(C)=(3.35\pm0.02)\quad\quad\quad \text{e}\quad\quad\quad b'=q'=(-1.006\pm0.006)$ 
\end{center}
dove $q'$ risulta compatibile col valore atteso.
\bigskip

Una volta fatto ciò è stato possibile calcolare la velocità di fase a partire dal valore $a'=ln(C)$ ottenuto nella regressione lineare con confidenza al $95\%$, determinando così $C=(28.5\pm 0.6)\ \frac{m}{s}$.
\bigskip

Poi si è verificato che effettivamente le unità di misura della velocità di fase $C$, ottenute dalla relazione $\lambda=Cf^q$, fossero quelle di una velocità, quando $q=-1$ (vedi Sezione \ref{sez: verifica unità di misura}).
\bigskip

Si è inoltre calcolata la velocità di fase tramite il Grafico lineare $\lambda(\frac{1}{f})$, dove $C'$ è proprio la pendenza della retta ottenuta. Il valore è $C'=(28.23\pm0.19)\ \frac{m}{s}$, compatibile col valore $C$ trovato in precedenza dalla regressione lineare.







\newpage

\section{Tabella utile ai fini dei calcoli}
{
\renewcommand\arraystretch{1.08} %spaziatura tra le colonne (da 1 a 1.1)
\begin{table}[h!] %TABELLA COMPLETA
    \centering
    \begin{tabular}{|c|c|c|c|c|} % numero di colonne
    
    \hline %LINEA DEI TITOLI
    \textbf{Modo} & \textbf{Frequenza} & \textbf{Distanza tra due nodi} & \textbf{Distanza media} & \textbf{Lunghezza d'onda} \\ (n) & ($f\pm\Delta f$)($s^{-1}$) & ($d_n\pm\Delta d_n$ )($m$) & ($\overline{d}\pm\overline{\Delta{d}}$)($m$) & ($\lambda\pm\Delta\lambda$ )($m$) \\ [1ex] 
    
    \hline\hline %LINEE DEI DATI
1 & 7.0$\pm$0.1  & 2.007$\pm$0.015 & 2.007$\pm$0.015 & 4.014$\pm$0.030 \\ [0.7ex] %ampiezza tra celle
    
    \hline %freq          %dist.inter.
2 & 14.0$\pm$0.1 & 1.005$\pm$0.008 & 1.005$\pm$0.008 & 2.010$\pm$0.016 \\ [0.7ex]
       &              & 1.005$\pm$0.013 &                 &            \\ [0.7ex]
    
    \hline
       &              & 0.665$\pm$0.008 &                 &            \\ [0.7ex]
3 & 20.9$\pm$0.1 & 0.670$\pm$0.005 & 0.670$\pm$0.005 & 1.340$\pm$0.010 \\ [0.7ex]
       &              & 0.680$\pm$0.012 &                 &            \\ [0.7ex]
    
    \hline 
       &              & 0.500$\pm$0.008 &                 &            \\ [0.7ex]
4 & 27.8$\pm$0.1 & 0.508$\pm$0.005 & 0.505$\pm$0.005 & 1.010$\pm$0.010 \\ [0.7ex]
       &              & 0.505$\pm$0.005 &                 &            \\ [0.7ex]
       &              & 0.505$\pm$0.013 &                 &            \\ [0.7ex]
   
    \hline %freq             dist.inter.
       &              & 0.401$\pm$0.008 &                 &            \\ [0.7ex]
       &              & 0.411$\pm$0.005 &                 &            \\ [0.7ex]
5 & 34.7$\pm$0.1 & 0.407$\pm$0.004 & 0.404$\pm$0.004 & 0.808$\pm$0.008 \\ [0.7ex]
       &              & 0.395$\pm$0.005 &                 &            \\ [0.7ex]
       &              & 0.407$\pm$0.013 &                 &            \\ [0.7ex]
     
    \hline
       &              & 0.330$\pm$0.006 &                 &            \\ [0.7ex]
       &              & 0.333$\pm$0.003 &                 &            \\ [0.7ex]
6 & 41.7$\pm$0.1      & 0.336$\pm$0.005 & 0.333$\pm$0.002 & 0.666$\pm$0.004 \\ [0.7ex]
       &              & 0.335$\pm$0.004 &                 &            \\ [0.7ex]
       &              & 0.332$\pm$0.002 &                 &            \\ [0.7ex]
       &              & 0.339$\pm$0.011 &                 &            \\ [0.7ex]
    
    \hline %freq           dist inter 
       &              & 0.288$\pm$0.006 &                 &            \\ [0.7ex]
       &              & 0.288$\pm$0.003 &                 &            \\ [0.7ex]
       &              & 0.291$\pm$0.004 &                 &            \\ [0.7ex]
7 & 48.7$\pm$0.1      & 0.285$\pm$0.004 & 0.285$\pm$0.003 & 0.570$\pm$0.006 \\ [0.7ex]
       &              & 0.277$\pm$0.004 &                 &            \\ [0.7ex]
       &              & 0.282$\pm$0.003 &                 &            \\ [0.7ex]
       &              & 0.290$\pm$0.011 &                 &            \\ [0.7ex]
    \hline

    \hline
    \end{tabular}
    \caption{\small {\textit{Tabella raffigurante le grandezze utili ai fini dei calcoli: Colonna\begin{footnotesize}{\textit{(1)}}\end{footnotesize}: numero di modi \textbf{-} Colonna \begin{footnotesize}{\textit{(2)}}\end{footnotesize}: frequenza \textbf{-} 
 Colonna \begin{footnotesize}{\textit{(3)}}\end{footnotesize}: distanza tra i nodi \textbf{-} Colonna \begin{footnotesize}{\textit{(4)}}\end{footnotesize}: distanza media tra i nodi \textbf{-} Colonna \begin{footnotesize}{\textit{(5)}}\end{footnotesize}: lunghezza d'onda. } }
         }
         
    
   
 \label{tab:Tabella_completa}
 \end{table}
}





\end{document}
