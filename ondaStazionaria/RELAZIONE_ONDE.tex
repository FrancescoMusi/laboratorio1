
\documentclass[12pt, a4paper]{article}
\usepackage{graphicx}
\usepackage{mathtools}
\usepackage{amsmath}
\usepackage{geometry}
\linespread{1.1}
\geometry{
 total={170mm,257mm},
 left=20mm,
 top=15mm,
 bottom=28mm
 }
 
 \title{\textbf{\scalebox{1.4}{\text{{\Huge Onde Stazionarie}}}}}
 \date{}

\author{\begin{small}Mussini Simone, Ruscillo Fabio, Musi Francesco\end{small}}



\begin{document}
\maketitle
\section{Obbiettivo}
Lo scopo di questa esperienza è quello di trovare la velocità di propagazione di un'onda elastica trasversale, attraverso la misura della frequenza e della lunghezza d'onda di varie onde stazionarie.\

L'onda stazionaria è un'interferenza di due onde contrarie con la stessa frequenza e si può generare dalla riflessione di un'onda.
L'equazione dell'onda stazionaria è:
\begin{equation*}
    y=\frac{A}{2}\sin{(kx - \omega t)} + \frac{A}{2}\sin{(kx + \omega t)}
\end{equation*}


\section{Materiale}
\begin{itemize}
\setlength\itemsep{0mm}
    \item Fune omogenea
    \item Sostegni
    \item Peso da 5 $kg$
    \item Generatore di segnale
    \item Pistone
    \item Metro a nastro da 2,10 $m$
\end{itemize}

\section{Procedimento}
Il set-up era montato in modo che ci fosse una corda \textbf{\textit{tesa}} legata a un sostegno fisso da un lato, e dall'altro, ad un peso da $5$ $kg$. Il pistone era posto in prossimità  dell'estremo fisso, in modo tale da limitare l'errore sulla misura della distanza tra gli due nodi.\ 

Abbimo realizzato la \textit{Tabella \ref{tab:Tabella_completa}} nel quale riportare i dati utili, coi rispettivi errori.\
  
\subsection{Frequenza e distanza tra gli internodi}
Per avere una prima stima delle frequenze di risonanza, abbiamo acceso il generatore di segnale collegato al pistone, il quale ha cominciato a generare delle oscillazioni sulla corda.\ 

Abbiamo iniziato a modulare la frequenza e l'intensità del segnale, fino ad ottenere un'onda stazionaria con ampiezza maggiore possibile.\

Dopo aver ripetuto l'operzione ottenendo diversi modi, abbiamo calcolato con maggiore precisione la frequenza di risonanza del modo fondamentale, i cui multipli ci permettono di individuare il range delle frequenze dei modi successivi, considerando come errore sulle frequenze quello strumentale, poichè era l'unica fonte di errore rilevante.\
 
Infine abbiamo calcolato la distanza tra due nodi successivi e l'errore su ogni singolo nodo, così da poter fare una misura più accurata sull'errore da attribuire alla distanza degli internodi, stando attenti nell'assegnare un errore maggiore agli estremi della corda, visto che i nodi, erano meno individuabili. Abbiamo riportando sia le frequenze che le distanze tra due nodi successivi, con i rispettivi errori, sulla Tabella \ref{tab:Tabella_completa} (vedi Colonna \begin{footnotesize}{\textit{(2)}}\end{footnotesize}: frequenza, Colonna \begin{footnotesize}{\textit{(3)}}\end{footnotesize}: distanza tra i nodi) .\


\subsection{Distanza media degli internodi e Lunghezza d'onda}
Poichè la distanza tra due nodi e il rispettivo errore, variano da internodo a internodo, utilizziamo la media pesata per il calcolo della distanza media, e dell'errore medio.\ 

Per il calcolo sulla distanza media usiamo la formula:
\begin{equation*}
\begin{aligned}
  &\overline{d} =\frac{\sum w_n\cdot d_n}{\sum w_n}
  &\quad{} 
  \end{aligned}
  \begin{aligned}
  & & & & & & & & &\text{dove il peso di ogni misura è:}& w_n=\frac{1}{\delta d_n^2}
  &
  \end{aligned}
\end{equation*}
Invece l'errore medio è calcolato come:
\begin{equation*}
    \delta\overline{d} =\frac{1}{\sqrt{\sum w_n}}
\end{equation*}
Otteniamo quindi i seguenti valori:
{
\renewcommand\arraystretch{1.2} %spaziatura tra le colonne (da 1 a 1.1)

\begin{table}[ht] %[hl] serve per mettere la tabella nel mezzo del testo


\begin{tabular}{|c|c|c|c|c|c|c|c|} 
 
 \hline
  Modi (n) & 1 & 2 & 3 & 4 & 5 & 6 & 7\\
  
\hline

  
  $\overline{d}\pm\delta\overline{d}$ ($m$) &\footnotesize{2.007$\pm$0.015}  &\footnotesize{1.005$\pm$0.007} &\footnotesize{0.670$\pm$0.004}&\footnotesize{0.505$\pm$0.003}&\footnotesize{0.404$\pm$0.002} &\footnotesize{0.333$\pm$0.001}&\footnotesize{0.285$\pm$0.001}\\
\hline


\end{tabular}\\
\caption{\small{\textit{Misura della distanza media tra gli internodi, con errore medio.} }}
    \label{tab:Error_MediapesataFalse}
\end{table}
}
L'errore ottenuto facendo la media pesata è \textbf{\textit{inferiore}} all'errore rilevato, quindi teniamo come errore sul valor medio della distanza tra due nodi successivi, il più piccolo tra quelli da noi misurati (vedi Tabella \ref{tab:Tabella_completa}, Colonna \begin{footnotesize}{\textit{(4)}}\end{footnotesize}: distanza media tra i nodi).




Successivamente siamo andati a calcolare la lunghezza d'onda, moltiplicando per due la distanza media e l'errore medio, poichè la distanza tra due nodi è la metà di una lunghezza d'onda:
\begin{equation*}
    \lambda=2\cdot\overline{d}
\end{equation*}
\begin{equation*}
    \delta\lambda=2\cdot\delta\overline{d}
\end{equation*}

 I risultati ottenuti sono riportati di seguito: 
 
 {
\renewcommand\arraystretch{1.2} %spaziatura tra le colonne (da 1 a 1.1)

\begin{table}[ht] %[hl] serve per mettere la tabella nel mezzo del testo


\begin{tabular}{|c|c|c|c|c|c|c|c|} 
 
 \hline
  Modi (n) & 1 & 2 & 3 & 4 & 5 & 6 & 7\\
  
\hline

  
  $\lambda\pm\delta\lambda$ ($m$) &\footnotesize{4.014$\pm$0.030}  &\footnotesize{2.010$\pm$0.016} &\footnotesize{1.340$\pm$0.010}&\footnotesize{1.010$\pm$0.010}&\footnotesize{0.808$\pm$0.008} &\footnotesize{0.666$\pm$0.004}&\footnotesize{0.570$\pm$0.006}\\
\hline


\end{tabular}\\
\caption{\small{\textit{Lunghezza d'onda di ogni modo} }}
    \label{tab:Lunghezza_d'onda}
\end{table}
}





\newpage

\section{Tabella Completa}
{
\renewcommand\arraystretch{1.08} %spaziatura tra le colonne (da 1 a 1.1)


\begin{table}[h!] %TABELLA COMPLETA
    \centering
    \begin{tabular}{|c|c|c|c|c|} % numero di colonne
    
    \hline %LINEA DEI TITOLI
    \textbf{Modo} & \textbf{Frequenza} & \textbf{Distanza tra due nodi} & \textbf{Distanza media} & \textbf{Lunghezza d'onda} \\ (n) & ($f\pm\delta f$)($Hz$) & ($d_n\pm\delta d_n$ )($m$) & ($\overline{d}\pm\delta\overline{d}$)($m$) & ($\lambda\pm\delta\lambda$ )($m$) \\ [1ex] 
    
    \hline\hline %LINEE DEI DATI
1 & 7.0$\pm$0.1  & 2.007$\pm$0.015 & 2.007$\pm$0.015 & 4.014$\pm$0.030 \\ [0.7ex] %ampiezza tra celle
    
    \hline %freq          %dist.inter.
2 & 14.0$\pm$0.1 & 1.005$\pm$0.008 & 1.005$\pm$0.008 & 2.010$\pm$0.016 \\ [0.7ex]
       &              & 1.005$\pm$0.013 &                 &            \\ [0.7ex]
    
    \hline
       &              & 0.665$\pm$0.008 &                 &            \\ [0.7ex]
3 & 20.9$\pm$0.1 & 0.670$\pm$0.005 & 0.670$\pm$0.005 & 1.340$\pm$0.010 \\ [0.7ex]
       &              & 0.680$\pm$0.012 &                 &            \\ [0.7ex]
    
    \hline 
       &              & 0.500$\pm$0.008 &                 &            \\ [0.7ex]
4 & 27.8$\pm$0.1 & 0.508$\pm$0.005 & 0.505$\pm$0.005 & 1.010$\pm$0.010 \\ [0.7ex]
       &              & 0.505$\pm$0.005 &                 &            \\ [0.7ex]
       &              & 0.505$\pm$0.013 &                 &            \\ [0.7ex]
   
    \hline %freq             dist.inter.
       &              & 0.401$\pm$0.008 &                 &            \\ [0.7ex]
       &              & 0.411$\pm$0.005 &                 &            \\ [0.7ex]
5 & 34.7$\pm$0.1 & 0.407$\pm$0.004 & 0.404$\pm$0.004 & 0.808$\pm$0.008 \\ [0.7ex]
       &              & 0.395$\pm$0.005 &                 &            \\ [0.7ex]
       &              & 0.407$\pm$0.013 &                 &            \\ [0.7ex]
     
    \hline
       &              & 0.330$\pm$0.006 &                 &            \\ [0.7ex]
       &              & 0.333$\pm$0.003 &                 &            \\ [0.7ex]
6 & 41.7$\pm$0.1 & 0.336$\pm$0.005 & 0.333$\pm$0.002 & 0.666$\pm$0.004 \\ [0.7ex]
       &              & 0.335$\pm$0.004 &                 &            \\ [0.7ex]
       &              & 0.332$\pm$0.002 &                 &            \\ [0.7ex]
       &              & 0.339$\pm$0.011 &                 &            \\ [0.7ex]
    
    \hline %freq           dist inter 
       &              & 0.332$\pm$0.002 &                 &            \\ [0.7ex]
       &              & 0.332$\pm$0.002 &                 &            \\ [0.7ex]
       &              & 0.332$\pm$0.002 &                 &            \\ [0.7ex]
7 & 48.7$\pm$0.1 & 0.332$\pm$0.002 & 0.285$\pm$0.003 & 0.570$\pm$0.006 \\ [0.7ex]
       &              & 0.332$\pm$0.002 &                 &            \\ [0.7ex]
       &              & 0.332$\pm$0.002 &                 &            \\ [0.7ex]
       &              & 0.332$\pm$0.002 &                 &            \\ [0.7ex]
    \hline

    \hline
    \end{tabular}
    \caption{\small {\textit{Tabella raffigurante le varie grandezze misurate e calcolate, coi rispettivi errori: Colonna\begin{footnotesize}{\textit{(1)}}\end{footnotesize}: numero di modi \textbf{-} Colonna \begin{footnotesize}{\textit{(2)}}\end{footnotesize}: frequenza \textbf{-} 
 Colonna \begin{footnotesize}{\textit{(3)}}\end{footnotesize}: distanza tra i nodi \textbf{-} Colonna \begin{footnotesize}{\textit{(4)}}\end{footnotesize}: distanza media tra i nodi \textbf{-} Colonna \begin{footnotesize}{\textit{(5)}}\end{footnotesize}: lunghezza d'onda. } }
         }
    
   
 \label{tab:Tabella_completa}
 \end{table}
}





\end{document}
