\documentclass[12pt, a4paper]{article}
\usepackage{graphicx}
\usepackage{mathtools}
\usepackage{xcolor}
\usepackage{amsmath}
\usepackage{geometry}
\graphicspath{ {./immagini/} }
\linespread{1.1}
\geometry{
 total={170mm,257mm},
 left=20mm,
 top=15mm,
 bottom=28mm
 }
 
 \title{\textbf{\scalebox{1.3}{\text{Moto di un grave lungo la verticale}}}}
 \date{}

\author{\begin{small}Mussini Simone, Ruscillo Fabio, Musi Francesco\end{small}}



\begin{document}
\maketitle
\section{Obbiettivo}
Lo scopo è verificare la grandezza \textit{g}, accelerazione di gravità terrestre. 
Ciò è stato fatto prendendo misure tramite fotoregistrazioni ad alta velocità di una sferetta metallica lasciata libera di cadere lungo la verticale. 
La formula da verificare è:
\begin{equation*}
    y(t) = y(0) + v(0)t + \frac{1}{2}gt^2
\end{equation*}


\section{Strumenti}
\begin{itemize}
\setlength\itemsep{0mm}
    \item Programma di analisi "Tracker"
    \item Video con 1000 fotogrammi al secondo
\end{itemize}

\section{Procedimento}
Il setup del video comprende un asta metallica alla sommità della quale è posizionata una sferetta metallica ferma. L'asta presenta dei segni distanziati di 10cm. 
Ad un certo punto la sferetta viene rilasciata. Affianco all'asta è presente un cronometro con sensibilità al millesimo di secondo. 
Essendo il tutto filmato da una fotocamera (Sony DSC-RX100M4) che registra a 1000fps, ad ogni frame il cronometro avanza di un millesimo di secondo.

  
\subsection{Setting del programma di tracciamento}
Per iniziare, mediante la funzione "asta di calibrazione" si è posto uguale a 10cm la distanza tra 2 segni sull'asta, in modo che il programma potesse generare un metro calibrato con cui misurare la posizione del target. \\
Poi si è stabilito il frame di inizio del moto, che è risultato essere il 49esimo.  \\
La posizione è stata tracciata ogni 10 frame, considerando il target come puntiforme, ponendo il punto di massa nell'estremità rivolta verso il basso. Il metro calibrato è stato posto sulla schermata in modo che coincidesse con la posizione del punto di massa a target fermo.\\
In totale sono state prese 41 misure. 


\subsection{Gestione degli errori}
L'alta velocità di acquisizione della fotocamera è a scapito della quantità di luce catturata, quindi anche della nitidezza dell'immagine. Per questo è stato necessario stimare ogni grandezza tramite un errore. \\
Gli errori sono stati misurati in base al numero di pixel di transizione tra il colore del target e dello sfondo, e sono stati stimati misura per misura visto che la chiarezza di ogni frame era variabile. Tramite il metro di calibrazione, è stato posto: 1$px$ = 0.0005$m$, pertanto
\begin{equation*}
    \Delta y= (\Delta pixel \cdot 0.0005)m
\end{equation*}


\section{Analisi dei dati}
Il frame di inizio della caduta è il 49esimo, che corrisonde ad un tempo iniziale \textit{$t_i = 0.012s$}.
Abbiamo poi terminato lo studio del moto al 459esimo fotogramma, che corrisponde all'istante \textit{$t_f = 0.422s$}, per un tempo totale di caduta \textit{$\Delta t = 0.410s$}. 
Si è scelto l'ultimo istante di caduta come quello appena precedente al rimbalzo della pallina sul banco di lavoro.


\subsection{Grafico spazio-tempo}
Il primo grafico realizzato con il programma di analisi dati \textit{Igor Pro} è stato il grafico spazio-tempo, ossia quello della legge oraria, che dovrebbe seguire la formula indicata nel primo paragrafo (traiettoria parabolica con posizione e velocità iniziali nulle). 
Sulle ordiate compaiono i valori numerici di \textit{y(t)} associati ai relativi errori  \textit{$\Delta y$}. Sulle ordinete ci sono gli istanti di tempo \textit{t}. 

In prima analisi, i dati ottenuti tramite la regressione di potenza del secondo ordine (\textit{poli 3}), non sono compratibili con i valori teorici, fatta eccezione per $\frac{g}{2}$:


\renewcommand{\theenumii}{\roman{enumii}}
    \begin{enumerate}
    \itemsep0em 
        \item Posizione iniziale: $K_0 = -0.0010852 \pm 0.000739$ ($K_0atteso = 0$)
        \item Velocità iniziale:  $K_1 =  0.020001 \pm 0.0128$ ($K_1atteso = 0$)
        \item g/2:                $K_2 =  4.9298 \pm 0.0353$ ($K_2atteso = 4.9033$)
    \end{enumerate}

\bigskip

Per aumentare la tolleranza e rientrare nei valori attesi abbiamo creato una nuova equazioe isolando il termine in relazione  quadratica con il tempo, calcolando il relativo errore, in modo da \textcolor{red}{NON SO PERCHE}: 

\begin{equation}
    y_new = y - K_0 - K_1t ;  
\end{equation}
\begin{equation}
 \Delta y_new = \Delta y + \Delta K_0 + \sqrt{(\frac{\Delta K_1}{K_1})^2 + (\frac{\Delta t}{t})^2}
 \end{equation}


 \bigskip


Avendo inserito questi nuovi valori (\textit{Grafico 1}), sono risultati i seguenti coefficienti compatibili:
\renewcommand{\theenumii}{\roman{enumii}}
    \begin{enumerate}
    \itemsep0em 
        \item Posizione iniziale: $K_0' =  4.3447 \cdot 10^-5 \pm 0.0015$ ($K_0'atteso = 0$)
        \item Velocità iniziale:  $K_1' =  -0.0017052 \pm 0.0252$ ($K_1'atteso = 0$)
        \item g/2:                $K_2' =  4.9346 \pm 0.0706$ ($K_2'atteso = 4.9033$)
    \end{enumerate}

\begin{figure}
\centering
\includegraphics[width=150mm]{graph1.jpg}
\caption{Grafico 1}
\end{figure}


\section{Regressione}

\end{document}
