\documentclass[12pt, a4paper]{article}
\usepackage{graphicx}
\usepackage{mathtools}
\usepackage{amsmath}
\usepackage{geometry}
\linespread{1.1}
\geometry{
 total={170mm,257mm},
 left=20mm,
 top=15mm,
 bottom=28mm
 }
 
 \title{\textbf{\scalebox{1.3}{\text{Moto di un grave lungo la verticale}}}}
 \date{}

\author{\begin{small}Mussini Simone, Ruscillo Fabio, Musi Francesco\end{small}}



\begin{document}
\maketitle
\section{Obbiettivo}
Lo scopo è verificare la grandezza \textit{g}, accelerazione di gravità terrestre. 
Ciò è stato fatto prendendo misure tramite fotoregistrazioni ad alta velocità di una sferetta metallica lasciata libera di cadere lungo la verticale. 
La formula da verificare è:
\begin{equation*}
    y(t)=\frac{1}{2}gt^2 + y(0)
\end{equation*}


\section{Strumenti}
\begin{itemize}
\setlength\itemsep{0mm}
    \item Programma di analisi "Tracker"
    \item Video con 1000 fotogrammi al secondo
\end{itemize}

\section{Procedimento}
Il setup del video comprende un asta metallica alla sommità della quale è posizionata una sferetta metallica ferma. L'asta presenta dei segni distanziati di 10cm. 
Ad un certo punto la sferetta viene rilasciata. Affianco all'asta è presente un cronometro con sensibilità al millesimo di secondo. 
Essendo il tutto filmato da una fotocamera (Sony DSC-RX100M4) che registra a 1000fps, ad ogni frame il cronometro avanza di un millesimo di secondo.

  
\subsection{Setting del programma di tracciamento}
Per iniziare, mediante la funzione "asta di calibrazione" si è posto uguale a 10cm la distanza tra 2 segni sull'asta, in modo che il programma potesse generare un metro calibrato con cui misurare la posizione del target. \\
Poi si è stabilito il frame di inizio del moto, che è risultato essere il 49esimo.  \\
La posizione è stata tracciata ogni 10 frame, considerando il target come puntiforme, ponendo il punto di massa nell'estremità rivolta verso il basso. Il metro calibrato è stato posto sulla schermata in modo che coincidesse con la posizione del punto di massa a target fermo.\\
In totale sono state prese 41 misure. 


\subsection{Gestione degli errori}
L'alta velocità di acquisizione della fotocamera è a scapito della quantità di luce catturata, quindi anche della nitidezza dell'immagine. Per questo è stato necessario stimare ogni grandezza tramite un errore. \\
Gli errori sono stati misurati in base al numero di pixel di transizione tra il colore del target e dello sfondo, e sono stati stimati misura per misura visto che la chiarezza di ogni frame era variabile. Tramite il metro di calibrazione, è stato posto: 1$px$ = 0.0005$m$, pertanto
\begin{equation*}
    \Delta y= (\Delta pixel * 0.0005)m
\end{equation*}






\end{document}
