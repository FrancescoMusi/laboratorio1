\documentclass[12pt, a4paper]{article}
\usepackage{graphicx}
\usepackage{mathtools}
\usepackage{amsmath}
\usepackage{geometry}
\linespread{1.1}
\usepackage{caption}
\usepackage[italian]{babel}

\geometry{
 total={170mm,257mm},
 left=20mm,
 top=15mm,
 bottom=28mm
 }
\title{\textbf{Misura della costante elastica di una molla}}
\date{}
\author{\begin{small}Mussini Simone, Ruscillo Fabio, Musi Francesco\end{small}}
\begin{document}
\maketitle



\section{Richiami teorici}
La molla è un corpo elastico, capace di deformarsi se viene applicata una forza e di tornare sempre allo stesso allungamento in condizioni di equlibrio. 
Poichè tende a ritornare alla sua posizione di equilibrio, quando deformata esercita una forza, che segue la \textit{Legge di Hooke}: 
\begin{equation}
    F_{el} = -K(X-X_0)
\end{equation}
Dove $X-X_0$ è la deformazione linare, $K$ è la costante elastica propria di ogni molla e il segno meno 
indica che è una fornza di richiamo, in direzione opposta alla forza esterna applicata.
In condizioni ideali, quindi senza forze dissipative, l'equazione del moto di una molla è quella di 
un oscillatore armonico con pulsazione $\omega = \sqrt{\frac{K}{m}}$ ($m$ = massa del corpo collegato alla molla, la massa della molla è trascurabile). La soluzione dell'equazione differenziale è una funzione sinusoidale con ampiezza $A$ e fase iniziale $\phi$:
\begin{equation}
    F_el = m\ddot{x} = -Kx   \quad\xrightarrow{}\quad   \frac{d^2x}{dt^2} = -\omega^2x  \quad\xrightarrow{}\quad   x(t) = A cos(\omega t + \phi)
\end{equation}
Il periodo di una oscillazione risulta: $T = \frac{2\pi}{\omega} = 2\pi\sqrt{\frac{m}{K}}$.




\section{Obbiettivi}
Gli obbiettivi di questo esperimento sono: 
\begin{enumerate}
    \setlength\itemsep{0cm}
    \item Determinare la costante elastica di una molla $(K_s)$ con metodo statico
    \item Utilizzare la molla come dinamometro statico per la determinazione di una massa incognita $(M_i)$
    \item Determinare la costante elastica di una molla precompressa $(K_p)$ e la forza di precompressione con il metodo statico $(F_0)$
\end{enumerate}





\section{Strumenti}
    \begin{itemize}
    \setlength\itemsep{0cm}
        \item Masse calibrate e massa ignota
        \item Molla armonica e molla precompressa 
        \item Metro a nastro 
        \item Cronometro 
        \item Sostegno fisso 
    \end{itemize}



    
\section{Setup}
Il setup dell'esperimento è costituito da un piedistallo, la cui funzione è di sostenere due aste collegate tra loro tramite un morsetto, in modo da formare un supporto per sostenere le molle in posizione verticale.
All'asta messa in orizzontale rispetto al piano del tavolo, viene poi fissato un metro a nastro al fine di misurare le deformazioni che verranno indotte sulle molle. 





\section{Analisi preliminare}
In questa analisi, consideriano trascuarbili gli errori sulla massa $m$ e sull'accelerazione di gravità $g$, in quanto hanno imprecisioni di ordini di grandezza minori rispetto alle misure effettuete con il metro, pertanto in questi passaggi possono essere trascurate.\\

Per avere una stima sull'ordine di grandezza dell'errore su $K_s$, si è calcolato l'errore relativo della costante elastica della molla in modo statico. È stata svolta una singola misura: con una massa di $m=250g$ l'allungamento è di $(L-L_0)=0.095\pm 0.004$ $m$ pertanto, l'errore relativo su $K_s$, usando la formula sottostante (errore di un quoziente), risulta essere circa il $4\%$.
\begin{equation*}
    F_{el} = mg = -K_s(L-L_0) \quad \xrightarrow{} \quad   
    \frac{\Delta K_s}{K_s} = \frac{\Delta (L-L_0)}{L-L_0}
\end{equation*}
(L'incertezza sulla variazione di lunghezza è stata ottenuta sommando le incertezze sulle 2 misure della posizione iniziale e finale, ciascuna di $2mm$.) \\

Analogamente, si è calcolato l'errore relativo della costante elastica della molla in modo dinamico, utilizzando la rerlazione che lega ques'ultima al periodo. (L'errore è stato calcolato sempre come errore di un quoziente.) 
\begin{equation*}
    T = 2\pi\sqrt{\frac{m}{K}} \quad \xrightarrow{} \quad K = \frac{4m\pi^2}{T^2} \quad \xrightarrow{} \quad
    \frac{\Delta K_d}{K_d}=\frac{\Delta M}{M}+2\frac{\Delta T}{T}
\end{equation*}
Per avere un valore sul periodo e sulla relativa incertezza più corretti possibile, visto che le misure
dinamiche presentano errori casuali molto maggiori di quelle statiche, procediamo misurando 10 periodi per 
attutire l'errore dovuto ai riflessi umani, e prendendo 30 misure per ridurre gli errori casuali.
Su queste misure svolgiamo l'analisi di un campione gaussiano come segue:

\subsection{Ricerca di $T$ e $\Delta T$}
!!!!!!!!!!!!!!! TABELLA DATI E PROCEDIMENTO !!!!!!!!!!!!!!!!!!!!

otteniamo un errore relativo su $K_d$ del 2 \%.





\section{Misura della costante elastica con il metodo statico}
Questo step consiste nel misurare l'allungamento delle due molle soggette a differenti forze, generate dai contrappesi. I dati sono riportati di seguito in Tabella \ref{tab: Misure Statiche Non Precompresse} e Tabella \ref{tab: Misure Statiche Precompresse}:\\

\begin{table}[!htb]
    \begin{minipage}[t]{.5\linewidth}
    \centering
        \begin{tabular}{|c|c|}
        \hline
        $M$ $(kg)$&$L\pm \Delta L$ $(m)$\\
        \hline
        $0.050$ & $0.326$$\pm$$0.002$\\
        $0.100$ & $0.346$$\pm$$0.002$\\
        $0.150$ & $0.366$$\pm$$0.002$\\
        $0.200$ & $0.387$$\pm$$0.002$\\
        $0.250$ & $0.407$$\pm$$0.002$\\
        $0.300$ & $0.427$$\pm$$0.002$\\
        $0.350$ & $0.448$$\pm$$0.002$\\
        $0.400$ & $0.468$$\pm$$0.002$\\
        $0.450$ & $0.489$$\pm$$0.002$\\
        $0.500$ & $0.509$$\pm$$0.002$\\
        \hline
    \end{tabular}
    \captionsetup{width=7cm}
    \caption{Molla Non Precompressa. Sono riportati la massa e la posizione finale misurata in seguito alla deformazione.}

    \label{tab: Misure Statiche Non Precompresse}
    \end{minipage}
    \begin{minipage}[t]{.5\linewidth}
    \centering
        \begin{tabular}{|c|c|}
            \hline
            $M$ $(kg)$&$L\pm \Delta L$ $(m)$\\
            \hline
            $0.050$ & $0.203$$\pm$$0.003$\\
            $0.100$ & $0.203$$\pm$$0.003$\\
            $0.150$ & $0.203$$\pm$$0.003$\\
            $0.200$ & $0.208$$\pm$$0.003$\\
            $0.250$ & $0.224$$\pm$$0.003$\\
            $0.300$ & $0.242$$\pm$$0.003$\\
            $0.350$ & $0.259$$\pm$$0.003$\\
            $0.400$ & $0.276$$\pm$$0.003$\\
            $0.450$ & $0.295$$\pm$$0.003$\\
            $0.500$ & $0.313$$\pm$$0.003$\\
            \hline
        \end{tabular}
        \captionsetup{width=7cm}
        \caption{Molla Precompressa. Al di sotto di una massa pari a $0.150$ $kg$ non si è ossservata alcuna deformazione della molla.}
        
        \label{tab: Misure Statiche Precompresse}
    \end{minipage} 
\end{table}


L'errore sulle masse non è specificato perchè lo consideriamo unitario sull'ultima cifra decimale, 
in quanto i valori erano tabulati e non sono stati misurati per questo esperimento. 



\subsection{Molla Armonica (non precompressa) - metodo statico}
\label{Molla non precompressa, statico}
Dato che la legge di Hook $Mg=K_s\cdot(L-L_0)$ ci suggerisce che l'allungameto sia in relazione lineare con la forza applicata, abbiamo attuato una regressione lineare. 
Per la molla non precompressa la legge di cui trovare i coefficienti è: 
\begin{equation*}
    L=M\cdot\displaystyle\frac{g}{K_s}+L_0
    \quad \xrightarrow{} \quad Y = A + BX
\end{equation*} 
dove:
\begin{equation*}
 A= L_0 \quad , \quad B=\displaystyle\frac{g}{K_s} \quad , \quad y = L \quad e \quad X = m
\end{equation*}
\bigskip

La figura sottostante è il grafico ottenuto da un $quick fit$ eseguito con il programma di analisi dati, che rappresenta anche la retta dei minimi quadrati.\\
\begin{figure}[h]
    \includegraphics[width=150mm]{IMMAGINI/Graph Molla Arm Stat.jpg}
    \centering
    \caption{\textit{{\footnotesize{Grafico $L(M)$}: sul'asse asse \textit{y} sono riportati i valori della posizione finale misurata $L$ coi rispettivi errori, sull'asse \textit{x} la massa corrispettiva}}}
    \label{Grafico parabolico}
\end{figure}
\bigskip

Calcolando $K_s$ e il rispettivo errore $\Delta K_s$ come:
\begin{equation*}
    K_s= \frac{g}{B} \quad , \quad \Delta K_s = \left | \frac{g}{B}  \right | \cdot \Delta B 
\end{equation*}
Si ottiene $K_s = (24.1 \pm 0.2)$ $N m^{-1}$



\subsection{Molla Precompressa - metodo statico}
Nel caso di una molla precompressa, la legge che lega l'allungamento alla forza applicata differisce dalla legge di Hooke per un termine costante $\frac{F_0}{K_p}$ , dovuto alla forza $F_0$ di precompressione della molla.  
\\La legge su cui abbiamo eseguito la regressione lineare risulata quindi: 
\begin{equation}
    L = m \frac{g}{k_p} + (L_0 - \frac{F_0}{K_p}) 
\end{equation}
dove 
\begin{equation}
    A = L_0 + \frac{F_0}{K_p} \  \ \ \   e   \ \ \ \ B = \frac{g}{K_p}
\end{equation}
\\Calcolando $K_p$ e il rispettivo errore $\Delta K_p$ come : 
\begin{equation}
    K_p = \frac{g}{B} e \Delta K_p = \left | \frac{g}{B}  \right | \cdot \Delta B 
\end{equation}






\section{Misura della costante elastica con metodo dinamico }
Questo step consiste nel misurare il tempo impiegato dalla molla, alla quale è fissato un pesetto da $500kg$, a compiere dieci oscillazioni complete. Al fine di ricavare la costante $k_d$ dalla relazione: 
\begin{equation}
    T = 2\pi \cdot \sqrt{\frac{k_d}{m}}
\end{equation} 

Per avere una miglior stima del periodo, abbiamo misurato 30  volte il tempo relativo alle 10 oscillazioni, inserendoli con le relative frequenze di misura, nella Tabella \ref{tab: Misure Dinamiche} a fine paragrafo.  

Abbiamo controllato che il campione fosse compatibile con una distribuzione gaussiana.
Calcolando il valore medio $\overline{x}$ e la deviazione standar $\sigma_x$ tramite le formule: 
\begin{equation}
    \overline{x} = \frac{\sum_{i=1}^{N}x_i}{N} \quad , \quad
    \sigma_x = \sqrt{\frac{\sum_{i=1}^{N}(x_i-\overline{x})^2}{(N-1)}}
\end{equation}

\smallskip
Nel nostro caso $\overline{x} = 9,2773$ e $\sigma_x = 0,0921 $. 
Ora possiamo dividere le misure delle dieci oscilazioni in classi con il relativo numero valori osservati e riportarle  nella seguente tabella: 


\begin{table}[hbt!]
    \centering
    {\renewcommand\arraystretch{1.0} 
    \begin{tabular}{|c|c|}
    \hline
        \textbf{Dieci Periodi} & \textbf{Frequenza}\\
        $10T(s)$ & $f(n)$\\
    \hline
    $ 9.07 $ & $ 1 $ \\ 
    $ 9.12 $ & $ 1 $ \\
    $ 9.13 $ & $ 1 $ \\
    $ 9.15 $ & $ 1 $ \\ 
    $ 9.21 $ & $ 1 $ \\
    $ 9.22 $ & $ 1 $ \\
    $ 9.23 $ & $ 3 $ \\
    $ 9.25 $ & $ 1 $ \\
    $ 9.26 $ & $ 3 $ \\
    $ 9.27 $ & $ 1 $ \\
    $ 9.28 $ & $ 2 $ \\
    $ 9.29 $ & $ 1 $ \\
    $ 9.30 $ & $ 3 $ \\ 
    $ 9.31 $ & $ 2 $ \\
    $ 9.32 $ & $ 2 $ \\
    $ 9.35 $ & $ 2 $ \\
    $ 9.37 $ & $ 1 $ \\
    $ 9.38 $ & $ 1 $ \\
    $ 9.47 $ & $ 1 $ \\
    $ 9,50 $ & $ 1 $ \\ 
    \hline
    \end{tabular}}
    \caption{Misurazioni dinamiche.}
    \label{tab: Misure Dinamiche}
\end{table}


I dati sono stati divisi in 6 classi come mostrato in figura. 

!!!!!!!!!!!ISTOGRAMMA !!!!!!!!!!!!!!!

Visto che nessun dato è più distante di 2.5 deviazioni standard dal valor medio, abbiamo deciso di accettarli tutti. Inoltre, dato che le 2 classi agli estremi hanno soltanto 1 o 2 valori, possiamo includerli nelle classi più interne per averne un totale di 4. 

Tramite il test del chi-quadro, che mette in relazione i valori osservati ($O_k$) con i valori attesi secondo una distribuzione gaussiana perfetta ($E_k$), si può verificare che sia effettivamente un campione gaussiano e quindi usare le formule per il valore migliore e l'errore.
\begin{equation*}
    \chi^2 = \sum_{k=0}^N \frac{(O_k - E_k)^2}{E_k} = 0.552
\end{equation*}
Con N = 30, numero di misure. La gaussiana ha 3 vincoli e visto che le classi sono 4, i gradi di libertà risultano essere 1.

Con questi dati, cosultando la relativa tabella, la confidenza risulta essere tra il 50 ed il 75\%. Essendo minore del 90\%, possiamo cosiderare gaussiano il nostro campione e utilizzare la seguente formula per ottenere l'errore sul valor medio. 
\begin{equation*}
    \delta \overline{x} = \frac{\sigma_x}{\sqrt{N}} \approx 0.02s \quad \xrightarrow{} \quad 10T = (9.28 \pm 0.02)s
\end{equation*}
Dividendo tutto per 10, otteniamo la misura del singolo periodo con il relativo errore. Tuttavia non possiamo tenere come errore $0.02s / 10$, in quanto l'errore è minore dell'errore strumentale, quindi dobbiamo considerare quest'ultimo. 
\begin{equation*}
    T = (0.93 \pm 0.01)s
\end{equation*}

Ora tramite la formula inversa si calcola la costante $k_d$ e il relativo errore considerando l'errore di una potenza per una costante, dato che le masse hanno valori tabulati e possiamo considerarle senza incertezza. 
\begin{equation*}
    k_d = \frac{T^2 m}{4\pi^2} = 0.010954N/m \quad , \quad \delta k_d = \frac{m}{4\pi^2} * 2 \frac{\delta T}{T} * k_d = 
\end{equation*}






\newpage
\section{Determinazione di una massa incognita}
Questa misura consiste nell'appendere la massa e misurarne l'allungamento ($\Delta L$). Per calcolare il valore di $m_i$ bisogna usare la relazione lineare del paragrafo \ref{Molla non precompressa, statico}, in quanto anche questa è stata una misura statica. I coefficienti sono già stati trovati e riportati qui sotto. 
Il processo è stato svolto con la molla non precompressa, e sulla massa incognita compariva la lettera "A".
\begin{equation}
    L = A + Bm  \mbox{\hspace{0.5cm} quindi \hspace{0.5cm}}  m_i = \frac{L - A}{B}
\end{equation}\\
Con i valori: $A = (0.3052 \pm 0.0011)m$, $B = (0.407 \pm 0.004)\frac{m}{kg}$, $L = (0.517 \pm 0.002)m$, calcolando l'errore sommando le derivate parziali oltiplicate per gli errori relativi, si ottiene: 

\begin{equation} 
    m_i = (0.520 \pm 0.013)kg 
\end{equation}




\end{document}
